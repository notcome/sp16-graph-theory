\documentclass{homework}

\title{Homework 1}

\begin{document}
\maketitle

\begin{problem}{1}
%%%% %%%% %%%% %%%% %%%% %%%% %%%% %%%% %%%% %%%% %%%% %%%% %%%% %%%% %%%% %%%%|
Since $|E| = \frac{1}{2}\sum_{v \in V} \deg{v}$, we have
$|E| \geq \frac{1}{2}\sum_{v \in V} \delta(\Gamma)
= \frac{1}{2}|v| \delta(\Gamma)$, namely
$\frac{|E|}{|V|} \geq \frac{\delta(\Gamma)}{2}$.
Similarly,
$|E| \leq \frac{1}{2}\sum_{v \in V} \Delta(\Gamma)
= \frac{1}{2}|v| \Delta(\Gamma)$, namely
$\frac{|E|}{|V|} \leq \frac{\Delta(\Gamma)}{2}$. \QED

Consider a graph of two vertexes and an edge between them. Both equality holds
for such a graph.
\end{problem}

\begin{problem}{2}
%%%% %%%% %%%% %%%% %%%% %%%% %%%% %%%% %%%% %%%% %%%% %%%% %%%% %%%% %%%% %%%%|
This can be proven by induction.

First, one can consider a graph with only one vertex, which is necessarily
connected. 

Then, for a graph with $n$ vertexes, one can divide these vertexes into two 
disjoint subsets and slice the edges set accordingly.
From the induction hypothesis it is evident that each subset itself constitutes 
a connected graph. Consider the sub-graph with less vertexes. Such a sub-graph
can have at most $\frac{n}{2}$ when $\Gamma$ has even vertexes, and
$\frac{n-1}{2}$ when odd.
\begin{itemize}
%%%% %%%% %%%% %%%% %%%% %%%% %%%% %%%% %%%% %%%% %%%% %%%% %%%% %%%% %%%% %%%%|
\item If $\Gamma$ has even vertexes, from the premise we
know that each vertex has at least $\frac{n}{2}$ adjacent vertexes in
$\Gamma$, namely at least one vertexes in the other sub-graph is adjacent
to each vertex in this smaller sub-graph. Hence $\Gamma$ is necessarily
connected.
\item If $\Gamma$ has odd vertexes, then each vertex has at least $\frac{n}{2}$ 
adjacent vertexes in $\Gamma$, namely at least one vertexes in the other
sub-graph is adjacent to each vertex in this smaller sub-graph.
Hence $\Gamma$ is necessarily connected.
\end{itemize}
This completes the proof.\QED
\end{problem}

\begin{problem}{3}
\begin{enumerate}
%%%% %%%% %%%% %%%% %%%% %%%% %%%% %%%% %%%% %%%% %%%% %%%% %%%% %%%% %%%% %%%%|
\item For each two distinct vertex $v_1, v_2$, there could either be an edge
or not. There are $\frac{n(n-1)}{2}$ such distinct pairs, so
\begin{align*}
m_n &= 2^\frac{n(n-1)}{2}\\
m_1 &= 1\\
m_2 &= 2\\
m_3 &= 8\\
m_4 &= 64
\end{align*}

%%%% %%%% %%%% %%%% %%%% %%%% %%%% %%%% %%%% %%%% %%%% %%%% %%%% %%%% %%%% %%%%|
\item
The number of connected graphs are the number of all graphs subtracting
that of disconnected ones. Use $d_n$ to denote the number of disconnected
graphs. One can fix a vertex and divide the graph into two parts, namely
vertexes connected to this vertex and all the rest. Assume there are $k$
in the ``all the rest'' subset, the total number of disconnected graphs
of this situation would be
$\begin{pmatrix}n-1\\k\end{pmatrix}m_k c_{n-k}$. Therefore
$$d_n = \sum^{n-1}_{k=1}\begin{pmatrix}n-1\\k\end{pmatrix}m_k c_{n-k}$$
, and the total number of connected graphs is
\begin{align*}
c_n &= m_n - d_n \\
    &= m_n - \sum^{n-1}_{k=1}\begin{pmatrix}n-1\\k\end{pmatrix}m_k c_{n-k}\\
c_1 &= 1\\
c_2 &= 1\\
c_3 &= 4\\
c_4 &= 38
\end{align*}
\end{enumerate}
\end{problem}

\end{document}