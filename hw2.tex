\documentclass{homework}

\title{Homework 2}

\begin{document}
\maketitle

\begin{problem}{1}
%%%% %%%% %%%% %%%% %%%% %%%% %%%% %%%% %%%% %%%% %%%% %%%% %%%% %%%% %%%% %%%%|
Consider two maximal paths $P, Q$ that do not have a common vertex. Label P's
vertexes $p_1, \hdots, p_n$, and Q's $q_1, \hdots, q_m$. Since the graph
is connected, there exists $p_i$ and $q_j$ such that there is a walk from
$p_i$ and $q_j$ whose edges does not occur in any of these two paths.

Next, one can choose the longer path between the one from $p_1$ to $p_i$
and the one from $p_n$ to $p_i$. Call the path $P'$. Do the same for $Q$ and
obtain $Q'$. The length of $P'$, denoted as $|P'|$, is at least
$\frac{1}{2}|P|$. Similarly, $|Q'| \geq \frac{1}{2}|Q|$. Join $P'$ and $Q'$
by the walk from $p_i$ to $q_j$. This new path's length is larger than
$\frac{1}{2}|P| + \frac{1}{2}|Q| = |P| = |Q|$, which means there exists a
path larger than $P$ or $Q$, a contradiction. Hence there do not exist
two maximal paths without a common vertex.

On the existence of such $p_i$ and $q_j$. Pick an arbitrary vertex in $P$
and another in $Q$. There exists a walk between these two vertexes. For
the sequence of vertexes of this walk, there must exists two vertexes, one
from $P$ and the other from $Q$, such that any vertex in between is not
in $P$ or $Q$. Such two vertexes are the claimed $p_i$ and $q_j$. \QED

The converse case does not hold, for one can consider a graph of two
disconnected complete sub-graphs with same amount of vertexes. One can
``replicate'' the maximal path found in one sub-graph in the other without
any common vertex.
\end{problem}

\begin{problem}{2}
%%%% %%%% %%%% %%%% %%%% %%%% %%%% %%%% %%%% %%%% %%%% %%%% %%%% %%%% %%%% %%%%|
Consider two complete sub-graphs, each of $2017$ vertexes. Add another vertex
that has an edge to every vertex in these two graphs. Label these two complete 
sub-graphs $\Gamma_1$ and $\Gamma_2$. Label the vertex $v$.

\textbf{Claim}. This is the graph $\Gamma$ desired.

\textbf{Proof}. Edge connectivity. $\lambda(\Gamma_1) = \lambda(\Gamma_2) 2016$.
If we cut some edges in $\Gamma_1$ (or $\Gamma_2$), there exists an edge between
some vertex in $\Gamma_1$ (or $\Gamma_2$) and $v$. Besides, it takes at least
$2017$ edge cuts on $v$ to disconnect $\Gamma_1$ (or $\Gamma_2$) from the rest
of $\Gamma$. Hence, $\lambda(\Gamma) = 2016$.

Vertex connectivity. Remove $v$. Hence, $\kappa(\Gamma) = 1$. \QED
\end{problem}

\begin{problem}{3}
%%%% %%%% %%%% %%%% %%%% %%%% %%%% %%%% %%%% %%%% %%%% %%%% %%%% %%%% %%%% %%%%|
Consider two complete sub-graphs, each of $5$ vertexes. Pick two vertexes from
each sub-graph, label them $v_1, v_2, u_1, u_2$---the two with the same alphabet
come from the same sub-graph. Add the following edges:
$\{v_1, u_1\}$, $\{v_2, u_2\}$, $\{v_1, u_2\}$.

\textbf{Claim}. This is the graph $\Gamma$ desired.

\textbf{Proof}. Edge connectivity. Cutting the three edges can disconnect
$\Gamma$. Cutting edges other than these edges requires four cuts
to disconnect since each of these two components are complete graphs.
Therefore, $\lambda(\Gamma) = 3$.

Vertex connectivity. Remove $v_1$ and $v_2$ will disconnect $\Gamma$.
Therefore, $\kappa(\Gamma) = 2$.

Since we construct the graph based on two complete sub-graphs of $5$ vertexes,
$\delta(\Gamma) = 4$. \QED
\end{problem}

\begin{problem}{4}
%%%% %%%% %%%% %%%% %%%% %%%% %%%% %%%% %%%% %%%% %%%% %%%% %%%% %%%% %%%% %%%%|
\end{problem}


\end{document}